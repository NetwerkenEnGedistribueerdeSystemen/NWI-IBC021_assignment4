\documentclass[12pt, a4paper]{article}

\usepackage{enumerate}
\usepackage{listings}
\usepackage{graphicx}

\lstset{
	breaklines=true
}

\setlength\parskip{1em}
\setlength\parindent{0em}

\title{Assignment 4}

\author{Hendrik Werner s4549775}

\begin{document}
\maketitle

\section{} %1
\includegraphics[width=\linewidth]{screenshots/dhcp}

\begin{enumerate}[a]
	\item %a
	The Ethernet address of my host is 18:cf:5e:15:89:47. It can be found in the Ethernet frame.

	\item %b
	The DHCP request is almost the same as the DHCP discover message except that it contains a DHCP Server Identifier with the IP of the DHCP server.

	\item %c
	The DHCP server's IP is 131.174.117.20.

	\item %d
	The lease time is $3600$ (seconds) = 1 hour. It can be found in the ACK messages.

	\item %e
	\begin{tabular}{|c|c|}
		\hline
		Action & Transaction ID\\\hline
		Release 1 & 0x61d63533\\\hline
		Request 1 & 0x7167c369\\\hline
		Release 2 & 0x70ca5e67\\\hline
		Request 2 & 0xdb284048\\\hline
	\end{tabular}

	Each transaction (set of messages that belong together) has an ID to keep track of the transaction should there be more than one.

	\item %f
	The release message is sent by the client to cancel the lease on its IP. The message is not acknowledged or retransmitted. If the DHCP server does not get this message then the lease will just run out after the lease time is exceeded. It is therefore not important to insure delivery of the release message.
\end{enumerate}

\section{} %2
\begin{enumerate}
	\item %1
	\begin{tabular}{|c|c|}
		\hline
		Interface & IP\\\hline
		A & 192.168.1.1\\\hline
		B & 192.168.1.2\\\hline
		C & 192.168.2.1\\\hline
		D & 192.168.2.2\\\hline
		E & 192.168.3.1\\\hline
		F & 192.168.3.3\\\hline
	\end{tabular}

	\item %2
	To answer this question we need to know the MAC addresses of the interfaces on the routers. I assume addresses $X_1$, $X_2$ for the right router and $Y_1$, $Y_2$ for the left router.

	\begin{enumerate}
		\item E sends a datagram to (192.168.1.2, $X_2$)
		\item $X_2$ forwards the datagram to (192.168.1.2, $Y_2$)
		\item $Y_2$ forwards the datagram to (192.168.1.2, B)
	\end{enumerate}

	\item %3
	In addition to the MAC addresses the routers now need IPs. I assume router X has IP 192.168.3.3 and router Y has IP 192.168.1.3.

	\begin{enumerate}
		\item E sends an ARP query to (192.168.3.3, FF-FF-FF-FF-FF-FF)
		\item The router responds with it's MAC address $X_2$ to (192.168.3.1, E)
		\item E receives the ARP response and knows the MAC address of the router. It sends the datagram to (192.168.1.2, $X_2$)
		\item $X_2$ forwards the datagram to (192.168.1.2, $Y_2$)
		\item $Y_2$ forwards the datagram to (192.168.1.2, B)
	\end{enumerate}
\end{enumerate}

\section{} %3
I assume the labels correspond to the MAC addresses, so that node A has MAC address A. I also assume these IPs and Switch Ports:

\begin{tabular}{|c|c|c|}
	\hline
	\textbf{Node} & \textbf{Port} & \textbf{IP}\\\hline
	A & 1 & 1.0.0.1 \\\hline
	B & 2 & 1.0.0.2 \\\hline
	C & 3 & 1.0.0.3 \\\hline
	D & 4 & 1.0.0.4 \\\hline
	E & 5 & 1.0.0.5 \\\hline
	F & 6 & 1.0.0.6 \\\hline
\end{tabular}

The switch table is empty initially.

\begin{tabular}{|c|c|c|}
	\hline
	\textbf{IP} & \textbf{MAC} & \textbf{Port}\\\hline
\end{tabular}

\begin{enumerate}[i]
	\item %i
	B sends a frame to E. The switch adds B to its switch table and floods the message to every port except port 2.

	\begin{tabular}{|c|c|c|}
		\hline
		\textbf{IP} & \textbf{MAC} & \textbf{Port}\\\hline
		1.0.0.2 & B & 2\\\hline
	\end{tabular}

	\item %ii
	Because the message from B was flooded. E received it and replies to B. The switch adds E to the switch table and sends the response to port 2 because B it in the switch table.

	\begin{tabular}{|c|c|c|}
		\hline
		\textbf{IP} & \textbf{MAC} & \textbf{Port}\\\hline
		1.0.0.2 & B & 2\\\hline
		1.0.0.5 & E & 5\\\hline
	\end{tabular}

	\item % iii
	A sends a frame to B. The switch adds A to its switch table. B is in the routing table so it forwards the message to port 2.

	\begin{tabular}{|c|c|c|}
		\hline
		\textbf{IP} & \textbf{MAC} & \textbf{Port}\\\hline
		1.0.0.2 & B & 2\\\hline
		1.0.0.5 & E & 5\\\hline
		1.0.0.1 & A & 1\\\hline
	\end{tabular}

	\item %iv
	B receives the frame from A and sends a response back. The switch already knows about B so nothing changes in the switch table and it knows A so it forwards the frame to port 1.

	\begin{tabular}{|c|c|c|}
		\hline
		\textbf{IP} & \textbf{MAC} & \textbf{Port}\\\hline
		1.0.0.2 & B & 2\\\hline
		1.0.0.5 & E & 5\\\hline
		1.0.0.1 & A & 1\\\hline
	\end{tabular}
\end{enumerate}

\section{} %4
Constantin Blach heroically copied the table from the exercise sheet into \LaTeX\ and I used this here. I filled it in myself however.

\begin{tabular}{c|c|c|c|c|c|c}
    ID & Source & Destination & Protocol & Content & Scenario 1 (a) & Scenario 2 (d) \\ \hline
    1 & C & broadcast & DHCP & DHCP & 1 & 1 \\ \hline
    2 & R & broadcast & ARP & who is $NS_A$ & - & - \\ \hline
    3 & A & C & HTTP & index.html & 6 & -\\ \hline
    4 & C & broadcast & ARP & who is R & 2 & 2 \\ \hline
    5 & R & $NS_A$ & DNS & query for A.com & 4 & - \\ \hline
    6 & C & A & TCP & SYN+ACK & 7 & 6 \\ \hline
    7 & C & A & HTTP & GET index.html & 5 & 4 \\ \hline
    8 & A & C & HTTP & Not Modified & - & 5 \\ \hline
    9 & C & R & DNS & query for A.com & 3 & 3 \\ \hline
    10 & R & broadcast & DHCP & DHCP offer & - & - \\ \hline
\end{tabular}

\begin{enumerate}
	\item %1
	Answer in the table.

	\item %2
	The packet with ID 2 is never sent because ARP is "point to point" and the message would never even reach $NS_A$.

	The packet with ID 10 is never sent because DHCP offers are not broadcast. The answer would be sent directly to C.

	\item %3
	The switch adds entries to its switch table the first time it receives a packet from them with their source MAC address. After the package with ID 1 it know the MAC of C, after R replies to the DHCP request the switch knows its MAC as well. This is not shown in the table.

	\item %4
	The ARP messages with ID 2 and 4 do not have a transport layer header because they operate on the Link Layer.

	\item %5
	Answer in the table.
\end{enumerate}

\section{} %5
\begin{enumerate}[a]
	\item %a
	I started the 3 hosts with these commands:
	\begin{lstlisting}
# terminal
vstart PC1 --eth0=A
vstart PC2 --eth0=A
vstart PC3 --eth0=A

# for PC1
ifconfig eth0 1.0.0.1 up
# for PC2
ifconfig eth0 1.0.0.2 up
# for PC3
ifconfig eth0 1.0.0.3 up
	\end{lstlisting}

	Then I ran arp on each machine. All ARP tables were empty:

	\includegraphics[width=\linewidth]{screenshots/screen1}

	Then I pinged PC2 from PC1 and ran arp again:

	\includegraphics[width=\linewidth]{screenshots/screen2}

	PC1 now knows about the MAC address of PC2 and vice versa. PC3 still has an empty ARP table.

	\item %b
	To ping PC2, PC1 needs to know PC2's MAC address. Because PC1's ARP table is empty it broadcasts an ARP query to the IP of PC2. PC2 receives the broadcast with PC1's IP and MAC and adds it to the ARP table. Next PC2 responds to the ARP query. PC1 receives PC2's response and adds an entry for PC2 to the ARP table.

	PC3 isn't involved in the transaction and does not get any updates. It receives the ARP broadcast from PC1 but does not respond to it because it does not match the IP of PC2.

	\item %c
	I spoofed PC2's MAC address from PC3:
	\begin{lstlisting}
arpspoof -i eth0 1.0.0.2
	\end{lstlisting}

	Next I pinged PC2 from PC2 and looked at the ARP tables again:

	\includegraphics[width=\linewidth]{screenshots/screen3}

	PC1 now has two entries (for PC2 and PC3) with the MAC address of PC2. The first entry is correct, the second one was spoofed by PC3.

	PC2 still has the entry for PC1 but now has an additional entry for PC3 but with the MAC of itself (PC2), that was spoofed by PC3.

	PC3 know has a correct ARP entry for PC2 because it looked up PC2's MAC address to spoof it.

	\paragraph{What happened?}
	PC3 started an ARP lookup for the IP 1.0.0.2 which belongs to PC2. It received the MAC address of PC2 and began sending out ARP responses with the MAC of PC2. PC2 and PC1 received those responses and added the spoofed entry to their ARP tables.
\end{enumerate}

\end{document}
